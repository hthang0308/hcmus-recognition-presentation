%!TeX encoding = UTF-8 Unicode
\documentclass{article}
\usepackage[pdftex]{graphicx} %for embedding images
\usepackage{url} %for proper url entries
\usepackage[bookmarks, colorlinks=false, pdfborder={0 0 0}, pdftitle={Research Outline - Recognition Spring 2021}, pdfauthor={Nhut-Nam Le}, pdfsubject={Recognition}, pdfkeywords={report, exercises}]{hyperref} %for creating links in the pdf version and other additional pdf attributes, no effect on the printed document
%\usepackage[final]{pdfpages} %for embedding another pdf, remove if not required
\usepackage[utf8]{inputenc}
\usepackage[vietnamese]{babel}
\usepackage{float}
\usepackage{fancyhdr}
\usepackage{pythonhighlight}
\usepackage[left=3cm, right=3cm, top=2cm, bottom=2cm]{geometry}
\usepackage{parskip}
\usepackage{tikz}
\usepackage{hyperref}
\usepackage[]{algorithm2e}
\usepackage[noend]{algpseudocode}
\usepackage{amsmath}
\usepackage{amsfonts}

\usepackage{listings}
\usepackage{color}

\definecolor{dkgreen}{rgb}{0,0.6,0}
\definecolor{gray}{rgb}{0.5,0.5,0.5}
\definecolor{mauve}{rgb}{0.58,0,0.82}

\newcommand\T{\rule{0pt}{2.6ex}}       % Top strut
\newcommand\B{\rule[-1.2ex]{0pt}{0pt}} % Bottom strut

\lstset{frame=tb,
	language=Java,
	aboveskip=3mm,
	belowskip=3mm,
	showstringspaces=false,
	columns=flexible,
	basicstyle={\small\ttfamily},
	numbers=none,
	numberstyle=\tiny\color{gray},
	keywordstyle=\color{blue},
	commentstyle=\color{dkgreen},
	stringstyle=\color{mauve},
	breaklines=true,
	breakatwhitespace=true,
	tabsize=3
}

\setlength{\parindent}{15pt}
\setlength{\headheight}{15.2pt}
\pagestyle{fancy}
\lhead[<even output>]{NHẬN DẠNG}
\rhead[<even output>]{ĐỀ CƯƠNG NGHIÊN CỨU}
\title{research-outline}
\author{Nhut-Nam Le}
\date{2021}

\usepackage{enumitem}
\setlistdepth{20}
\renewlist{itemize}{itemize}{20}

% initially, use dots for all levels
\setlist[itemize]{label=$\cdot$}

% customize the first 3 levels
\setlist[itemize,1]{label=\textbullet}
\setlist[itemize,2]{label=--}
\setlist[itemize,3]{label=*}
\begin{document}
	\begin{titlepage}
		\begin{center}
			% Top of the page
			\large{\textbf{ĐẠI HỌC KHOA HỌC TỰ NHIÊN, ĐHQG-HCM\\KHOA CÔNG NGHỆ THÔNG TIN\\BỘ MÔN KHOA HỌC MÁY TÍNH}}\\
			\includegraphics[width=0.75\textwidth]{images/khtn.png}\\
			% Title
			\huge \textbf{NHẬN DẠNG}\\[0.1in]
			\huge \textbf{ĐỀ CƯƠNG NGHIÊN CỨU}\\[0.1in]
			\vfill
			\normalsize
			% Submitted by
			\normalsize
			% Lecturers
			\textbf{Giảng viên lý thuyết}\\
			{\textbf{PGS. TS.} Lê Hoàng Thái}\\[0.1in]
			% Teacher Assistant
			\textbf{Giảng viên hướng dẫn}\\
			\vspace{0.1in}
			{Lê Thanh Phong, Nguyễn Ngọc Thảo}\\[0.1in]
			\textbf{Sinh viên thực hiện} \\
			\vspace{0.1in}
			{Nguyễn Hoàng Đức, Lê Nhựt Nam, Nguyễn Viết Dũng}\\[0.1in]
			% Date time when written report
			\vfill
			Tháng 4 năm 2021
		\end{center}
	\end{titlepage}
	\newpage
	% End Title4
	
	\pagenumbering{roman} %numbering before main content starts
	\cleardoublepage
	%\pagebreak
	
	\newpage
	\tableofcontents
	\newpage
	\pagenumbering{arabic} %reset numbering to normal for the main content
	\setcounter{secnumdepth}{0}
	
	\section{1. THÔNG TIN NHÓM}
	\begin{itemize}
		\item Thành viên 01: Nguyễn Hoàng Đức - 18120018
		\item Thành viên 02: Lê Nhựt Nam - 18120061
		\item Thành viên 03: Nguyễn Viết Dũng - 18120167
	\end{itemize}
	
	\section{2. THÔNG TIN ĐỀ TÀI}
	\subsection{Tên đề tài}
	\begin{itemize}
		\item Tên đề tài (Tiếng Việt): Sinh trắc học giọng nói
		\item Tên đề tài (Tiếng Anh): Voice Biometrics
	\end{itemize}
	\subsection{Nguồn tham khảo}
	\begin{itemize}
		\item Sách: Chương 8: Voice Biometrics, Handbook of Biometric, Anil K. Jain, Patrick Flynn, Arun A.
		Ross
		\item Paper with code: Speaker Recognition
	\end{itemize}
	Danh sách các bài báo khoa học tham khảo:
	\begin{itemize}
		\item Deep neural networks for small footprint text-dependent speaker
		verification, Ehsan Variani (Johns Hopkins Univ., Baltimore, MD, USA), Xin Lei (Google Inc.,
		USA), Erik McDermott (Google Inc., USA), Ignacio Lopez Moreno (Google Inc, Mountain View,
		CA, US), Javier Gonzalez-Dominguez (Google Inc., USA), 2014 IEEE International Conference on Acoustics, Speech and Signal
		Processing (ICASSP), Florence, Italy
		\item Multi-Task Learning for Text-Dependent Speaker Verification, Nanxin Chen, Yanmin Qian, Kai Yu (Shanghai Jiao Tong University, China), INTERSPEECH 2015 16th Annual Conference of the International Speech Communication Association, Dresden, Germany, 2015
		\item X-Vectors: Robust DNN Embeddings for Speaker Recognition, David Snyder, Daniel Garcia-Romero, Gregory Sell, Daniel Povey, Sanjeev Khudanpur (Center for Language and Speech Processing \& Human Language Technology Center of Excellence, The Johns Hopkins University, Baltimore, MD, USA), 2018 IEEE International Conference on Acoustics, Speech and Signal
		Processing (ICASSP), Calgary, AB, Canada, 2018
		\item Mirco Ravanelli and Yoshua Bengio. Speaker recognition from raw waveform with sincnet, 2019.
	\end{itemize}
	\subsection{Từ khóa}
	\begin{itemize}
		\item Tên từ khóa (Tiếng Việt): Nhận dạng sinh trắc học, Nhận dạng giọng nói, Nhận dạng người nói, Mạng Neural Tích chập, Mẫu thô, Xác minh người nói, Định danh người nói
		\item 	Tên từ khóa (Tiếng Anh): Biometric Recognition, Voice Recognition, Speaker Recognition, Convolutional Neural Networks, Raw Samples, Speaker Verification, Speaker Identification
	\end{itemize}
	
	\subsection{Nội dung trình bày}
	\qquad Nội dung trình bày trước lớp gồm 2 phần: trình bày nội dung tìm hiểu được từ chương 08 - Voice Biometrics sách Handbook of Biometric, trình bày những phương pháp state-of-the-art về lĩnh vực nhận dạng giọng nói trong thời gian gần đây.
	
	Phần trình bày nội dung tìm hiểu được từ sách
	\begin{itemize}
		\item Giới thiệu chung
		\item Những thông tin nhận dạng trong tín hiệu giọng nói
		\item Rút trích đặc trưng và phân tách
		\item Hai công nghệ chính của lĩnh vực nhận dạng giọng nói
	\end{itemize}
	Phần trình bày những phương pháp state-of-the-art về lĩnh vực nhận dạng giọng nói
	\begin{itemize}
		\item Giới thiệu
		\item Động lực nghiên cứu khoa học
		\item Phát biểu bài toán
		\item Các công trình tiêu biểu (khoảng 3 công trình)
		\begin{itemize}
			\item Deep neural networks for small footprint text-dependent speaker verification - Đại diện cho d-vectors (Deep Vectors)
			\item Multi-Task Learning for Text-Dependent Speaker Verification - Đại diện cho j-vectors
			\item X-Vectors: Robust DNN Embeddings for Speaker Recognition - Đại diện cho x-vectors
			\item Speaker recognition from raw waveform with sincnet - Đại diện cho Phân lớp người nói
		\end{itemize}
		\item So sánh d-vectors, j-vectors và x-vectors
		\item Demo
		\item Tài liệu tham khảo
	\end{itemize}
	\subsection{Nội dung báo cáo}
	\begin{itemize}
		\item Thông tin nhóm
		\item Thông tin đề tài
		\item Nội dung phân công
		\item Nội dung báo cáo
		\begin{itemize}
			\item Nội dung tìm hiểu từ sách Handbook of Biometric
			\begin{itemize}
				\item Giới thiệu chung
				\item Thông tin nhận dạng trong tín hiệu giọng nói
				\item Rút trích đặc trưng và phân tách thông tin
				\begin{itemize}
					\item Phân tích theo từng đoạn ngắn
					\item Tham số hóa
					\item Phân tách ngữ âm và tách từ
					\item Phân tách ngữ điệu
				\end{itemize}
				\item Công nghệ nhận dạng giọng nói phụ thuộc văn bản
				\item Công nghệ nhận dạng giọng nói không phụ thuộc văn bản
			\end{itemize}
			\item Các phương pháp SOTA
			\begin{itemize}
				\item Giới thiệu chung định hướng gần đây
				\item Động lực nghiên cứu
				\item Phát biểu bài toán: Đầu vào (Input), Đầu ra (Output)
				\item Kho ngữ liệu
				\item Kiểm định mô hình nhận dạng giọng nói
				\item Các độ đo thường dùng trong nhận dạng giọng nói
				\item Các công trình tiêu biểu
				\begin{itemize}
					\item Trong tác vụ rút trích đặc trưng
					\begin{itemize}
						\item d-vectors
						\item j-vectors
						\item x-vectors
						\item So sánh d-vectors, j-vectors và x-vectors
					\end{itemize}
					\item Trong phân lớp người nói
					\begin{itemize}
						\item Variational autoencoder
						\item Multi-domain features
						\item SincNet
					\end{itemize}
				\end{itemize}
			\end{itemize}
			\item Thực nghiệm: Sử dụng SincNet trong xác minh và định danh người nói
			\begin{itemize}
				\item Chuẩn bị dữ liệu
				\item Xây dựng mô hình
				\item Đăng ký
				\item Đánh giá
			\end{itemize}
		\end{itemize}
	\end{itemize}
	\subsection{Xây dựng demo cho chủ đề nghiên cứu}
	\begin{itemize}
		\item Phương pháp giải quyết vấn đề
		\begin{itemize}
			\item Dựa trên source code chính thức SincNet từ tác giả
			\item Huấn luyện mô hình xác minh người nói bằng tiếng Anh, tiếng Việt
		\end{itemize}
		\item Dữ liệu thực nghiệm
		\begin{itemize}
			\item TIMIT Acoustic-Phonetic Continuous Speech Corpus
			\begin{itemize}
				\item Tác giả: 	John S. Garofolo, Lori F. Lamel, William M. Fisher, Jonathan G. Fiscus, David S. Pallett, Nancy L. Dahlgren, Victor Zue
				\item Năm: 1993
				\item DCMI Type(s): âm thanh - sound
				\item Sample Type: 1-channel pcm (mono channel)
				\item Nguồn thu dữ liệu: microphone speech
				\item Ngôn ngữ: tiếng Anh
				\item Kích thước: 630 người nói từ 8 đặc trưng giọng Anh Mỹ, 1,3GB
			\end{itemize}
			\item Librispeech
			\begin{itemize}
				\item Tên đầy đủ: LibriSpeech ASR corpus
				\item Bài báo: LibriSpeech: an ASR corpus based on public domain audio books
				\item Tác giả/ Nhóm tác giả: Vassil Panayotov, Guoguo Chen, Daniel Povey and Sanjeev Khudanpur
				\item Được công bố tại hội nghị The international Conference on Acoustics, Speech, \& Signal Processing (ICASSP), 2015
				\item Mô tả sơ lược: Kho ngữ liệu có kích thước khoảng 1000 giờ nói tiếng Anh với tần số 16kHz
			\end{itemize}
			\item Son et al. Dataset
			\begin{itemize}
				\item Bài báo: Vietnamese Speaker Authentication Using Deep Models
				\item Dung lượng của tập dữ liệu: 535 MB
				\item Số mẫu trong tập dữ liệu: 400 mẫu
				\item Bộ dữ liệu gồm: hai tập  Men và Women, mỗi tập con chứa 10 thư mục người nói. Mỗi thư mục người nói chứa 20 đoạn ghi âm, chia ra Long và Short (mỗi loại 10 đoạn) 
				\item Nội dung câu nói
				\begin{itemize}
					\item Câu ngắn: “Tôi là sinh viên chuyên ngành công nghệ thông tin"
					\item Câu dài: "Tôi là sinh viên Học viện Công nghệ Bưu chính Viễn thông, chương trình đào tạo khá nặng đòi hỏi sinh viên phải học tập và nghiên cứu rất nhiều nhưng tôi tự hào vì đó là ngành đã và đang làm thay đổi cuộc sống xã hội loài người".
				\end{itemize}
				\item Điểm hạn chế: Bộ dữ liệu có kích thước khá nhỏ
			\end{itemize}
		\end{itemize}
		
		
		\item Thực nghiệm và đánh giá
		\begin{itemize}
			\item Trực quan hóa quá trình huấn luyện dựa trên các thông số: average training loss, classification error, average test loss, classification error frame level test data, classification error sentence level test data
			\item Dự đoán người nói từ mô hình
			\item Xác minh người nói bằng cosine similarity
		\end{itemize}
	\end{itemize}
	
	\section{3.	THÔNG TIN TỰ ĐÁNH GIÁ TIẾN ĐỘ}
	\subsection{Những công việc đã thực hiện được}
	Về những kiến thức chương 8 - sách Handbook of Biometrics
	\begin{itemize}
		\item Nắm cái nhìn chung về lĩnh vực Nhận dạng Giọng nói.
		\item Những yếu tố nhận dạng trong tín hiệu hiệu giọng nói.
		\item Hai công nghệ chính trong lĩnh vực nhận dạng giọng nói.
		\item Một số kỹ thuật thủ công trong xử lý một số đặc trưng nhận dạng tín hiệu giọng nói.
		\item Chuẩn bị slides thuyết trình cho phần này.
	\end{itemize}
	Về những kiến thức các phương pháp state-of-the-art trong lĩnh vực Nhận dạng Giọng nói
	\begin{itemize}
		\item Nắm sơ bộ những thành tựu gần đây trong lĩnh vực Nhận dạng Giọng nói.
		\item Nắm sơ bộ key methods, phương pháp, hiệu năng của một số phương pháp Deep Learning trong rút trích đặc trưng tín hiệu giọng nói.
		\item Nắm sơ bộ key methods, phương pháp, hiệu năng của một số phương pháp Deep Learning trong phân lớp tín hiệu giọng nói.
		\item Chuẩn bị slides thuyết trình cho phần này.
	\end{itemize}
	Về Tìm kiếm open source / libraries / demonstration để minh họa cho nội dung lý thuyết
	\begin{itemize}
		\item Tìm kiếm và đọc hiểu source code SincNet
		\item Quá trình hoạt động của các bộ lọc, kiến trúc mạng 
		\item Đã huấn luyện thành công mô hình cho tiếng Anh (train và test trên TIMIT, Librispeech) và tiếng Việt (train và test trên tập Son et al. Dataset)
		\item Đã kiểm thử similarity trên một vài file âm thanh đầu vào với tiếng Anh, tiếng Việt
	\end{itemize}
	\subsection{Những công việc chưa thực hiện được}
	Về những kiến thức chương 8 - sách Handbook of Biometrics
	\begin{itemize}
		\item Điều chỉnh slides thuyết trình
	\end{itemize}
	Về tìm kiếm open source / libraries / demonstration để minh họa cho nội dung lý thuyết
	\begin{itemize}
		\item Hoàn chỉnh 2 giai đoạn đầu tiên của một Speaker Recognition Pipeline (data preparation và development models, basic testing), chưa thực hiện enrollment, evaluation đầy đủ
		\item Số lượng epochs huấn luyện cho mô hình tiếng Anh, tiếng Việt còn khá thấp với (100 epochs với TIMIT và Librispeech, 300 epochs Son et al. Dataset)
	\end{itemize}
	\subsection{Hướng giải quyết khó khăn}
	\begin{itemize}
		\item Tiếp tục tìm hiểu và phát triển hoàn chỉnh Speaker Recognition Pipeline 
		\item Tìm kiếm một bộ dữ liệu cho tiếng Việt lớn hơn
	\end{itemize}
\end{document}